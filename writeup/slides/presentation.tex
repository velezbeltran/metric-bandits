\documentclass{beamer}
\usepackage[english]{babel}
\usepackage[utf8]{inputenc} % allow utf-8 input
\usepackage[T1]{fontenc}    % use 8-bit T1 fonts
\usepackage{hyperref}       % hyperlinks
\usepackage{url}            % simple URL typesetting
\usepackage{booktabs}       % professional-quality tables
\usepackage{amsfonts}       % blackboard math symbols
\usepackage{nicefrac}       % compact symbols for 1/2, etc.
\usepackage{microtype}      % microtypography
\usepackage{xcolor}         % colors
\usepackage{lipsum}
\usepackage{amsmath}
\usepackage{setspace}
\usepackage{bbm}
\usepackage[linesnumbered,ruled,commentsnumbered]{algorithm2e}
\usepackage{parskip}
\usepackage{graphicx}
\usepackage{caption}
\usepackage{subcaption}


%Information to be included in the title page:
\usetheme{Boadilla}
\usefonttheme[]{serif}
\title[Similarity Learning]{Active and Online Similarity Learning}
\subtitle[]{An application of bandits to Similarity learning}
\author[Ketan and Nicolas]{Ketan Jog and Nicolas Beltran}
\institute[]{Columbia University}
\date{May 2021}

\begin{document}

\begin{frame}
    \titlepage
\end{frame}


\begin{frame}
\frametitle{Long Term Goal:}
To reduce in a provable way online and active metric learning to a bandit problem (hopefully encoding hierarchical structure).
\end{frame}

% ----------------------------------------------------
% General summary of accomplishments
% ----------------------------------------------------
\begin{frame}{}
\frametitle{What we did:}
Use bandit algorithms to solve the following problems:
\begin{itemize}
\item
    Online learning of a similarity measure.
\item
    Active learning of a similiarity measure (i.e. by querying labels).
\end{itemize}
\vspace*{1cm}
\begin{definition}
Similarity Measure: A function $\phi: \mathbb{R}^{2n} \to \mathbb{R}$ that maps datapoints $x, y \in \mathbb{R}^{n}$ to a real number in $\mathbb{R}$
according to how ``similar'' they are.
\end{definition}
\end{frame}


% ----------------------------------------------------
% Problem formulation
% ----------------------------------------------------
\begin{frame}{}
    \frametitle{Problem: Online Similarity Learning}
    At round $t$ the environment samples $K$ pairs of points $(x_{t,k}, y_{t,k}) \in \mathbb{R}^{2n}$.
    We choose pair $k_t \in [K]$ and get reward $r_{t,k_{t}} \in \{1, -1\}$ based on whether they are similar.

    We are trying to minimize:
    \[ R_T = \mathbb{E}\left[\sum_{t =1}^T \phi(x_{t,k^\star_t}, y_{t,k^\star_t}) - \phi(x_{t,k_t}, y_{t,k_t})\right]\]
    where $k_t^\star = \text{argmax}_{k\in [K]} \phi(x_{t,k}, y_{t,k})$
\end{frame}

\begin{frame}{}
    \frametitle{Problem: Active Similarity Learning}
    Learner has access to a dataset $D = \{x_i \in \mathbb{R}^n| i \in [N]\}$ of unlabeled points.
    The learner can query $T$ pairs of points in this set $D$ to obtain a dataset $D_T = \{(x_t, y_t, r_t) ~|  ~t \in [T]\}$.
    The learner mantains and estimate $\hat{\phi}_t \in \mathcal{F}$ of $\phi$, where $\mathcal{F}$ is its function class.
    Denote the loss between an estimate $\hat{\phi}$ a $\phi$ as
    \[ \mathcal{L}(\phi, \hat{\phi}) = \mathbb{E}_{(x, y) \sim \mathcal{D} \times \mathcal{D}}[(\hat{\phi}(x,y) - \phi(x, y))^2] \]
    Goal is to find
    \[\min_{\phi \in \mathcal{F}} \mathcal{L}(\hat{\phi}_T, \phi)\]

\end{frame}

\end{document}
