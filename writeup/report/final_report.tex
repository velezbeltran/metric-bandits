\documentclass{article}
\usepackage[final]{neurips_2021}

\usepackage[utf8]{inputenc} % allow utf-8 input
\usepackage[T1]{fontenc}    % use 8-bit T1 fonts
\usepackage{hyperref}       % hyperlinks
\usepackage{url}            % simple URL typesetting
\usepackage{booktabs}       % professional-quality tables
\usepackage{amsfonts}       % blackboard math symbols
\usepackage{nicefrac}       % compact symbols for 1/2, etc.
\usepackage{microtype}      % microtypography
\usepackage{xcolor}         % colors
\usepackage{lipsum}
\usepackage{amsmath}
\usepackage{setspace}
\usepackage{bbm}
\usepackage[linesnumbered,ruled,commentsnumbered]{algorithm2e}
\usepackage{parskip}


\IncMargin{1.5em}

\title{Active Heirarchical Metric Learning}

\author{
  Nicolas Beltran\\
  Department of Computer Science\\
  Columbia University\\
  New York City, NY 10027 \\
  \texttt{nb2838@columbia.edu}\\
  \And
  Ketan Jog\\
  Department of Computer Science\\
  Columbia University\\
  New York City, NY 10027 \\
  \texttt{kj2473@columbia.edu}\\
}

\begin{document}

\maketitle

\begin{abstract}
    Many problems require a well defined notion of a distance between points in space.
    Constructing or finding such a measure falls into the field of metric learning.
    Although many algorithms exist in the field when a learner has access to a fixed dataset,  there is room for improvement in terms of samples efficiency that the learner needs to know, imposition of desired structure, especially when the data appears in an \textit{online} manner.
    We propose a project that reduces the problem of online/active metric learning to bandits. In case our plan turn out to be too ambitious, we have a fallback - an empirical investigation of some algorithms that have dealt with the problem in an online setting or in situations where the learner can selective query the points that it wants to know information about.
\end{abstract}


\section{Introduction}

\section{Related Work}

\section{Long-term goals}

\section{Preliminaries}


% --------------------------------------------------------------
% --------------------------------------------------------------
\section{Problem Statement}
\label{problem-statement}
% --------------------------------------------------------------
% --------------------------------------------------------------
We consider two different problems.
The first problem consists of making a series of sequential predictions while learning a
similiarity measure. We refer to this problem as online similarity prediction.
The second problem consists of learning a similarity measure while querying points in space.
We refer to this problem as active similarity learning.
A precise description of both problems is provided below.

\subsection{Online similarity learning}
\label{problem-statement:online-similarity-learning}
We consider an online similarity learning problem played over $T$ rounds.
At round $t$ the environment samples $K$ pairs of points $(\mathbf{x}_{t,k}^1, \mathbf{x}_{t,k}^2) \in \mathbb{R}^{2n}$.
The agent then chooses pair $k_t \in [K]$ and is given a reward $r_{t,k_{t}} \in \{1, -1\}$.
We assume that there exists some similarity function unknown to the agent $\phi: \mathbb{R}^{2n} \to \{-1, 1 \}$
and that the rewards are such that if at time $t \in [T]$ the agent chooses pair $(\mathbf{x}_{t,k}^1, \mathbf{x}_{t,k}^2)$
then the reward is $\phi(\mathbf{x}_{t,k}^1, \mathbf{x}_{t,k}^2)$.

As usual we define the regret as
\[ R_T = \mathbb{E}\left[\sum_{t =1}^T \phi(\mathbf{x}_{t,k^\star_t}^1, \mathbf{x}_{t,k^\star_t}^2) - \phi(\mathbf{x}_{t,k_t}^1, \mathbf{x}_{t,k_t}^2)\right]\]
where $k_t^\star = \text{argmax}_{k\in [K]} \phi(\mathbf{x}_{t,k}^1, \mathbf{x}_{t,k}^2)$

\subsection{Active similarity learning}
\label{problem-statement:active-similarity-learning}
We assume that the learner has access to a dataset $D = \{\mathbf{x}_i \in \mathbb{R}^n| i \in [N]\}$ of unlabeled points and that there exists some function $\phi: \mathbb{R}^{2n} \to \{-1, 1\}$ which the learner is trying to learn.
The learner can query $T$ pairs of points in this set $D$ to obtain a dataset $D_T = \{(\mathbf{x}_t^1, \mathbf{x}_t^2, y_t) ~|  ~t \in [T]\}$ . We assume that the learner mantains and estimate $\hat{\phi}_t \in \mathcal{F}$ of $\phi$, where $\mathcal{F}$ is its function class  and denote the loss between an estimate $\hat{\phi}$ a $\phi$ as
\[ \mathcal{L}(\phi, \hat{\phi}) = \mathbb{E}_\mathcal{(\mathbf{x}, \mathbf{y}) \sim \mathcal{D} \times \mathcal{D}}[(\hat{\phi}(\mathbf{x},\mathbf{y}) - \phi(\mathbf{x}, \mathbf{y}))^2] \]
It's objective is to find $\min_{\phi \in \mathcal{F}} \mathcal{L}(\hat{\phi}_T, \phi)$


\section{Description of the algorithms}
In total we provide 4 different algorithms which we describe below.
However, in reality they can be thought of as 2 different algorithms which slight modifications to accomodate for the online similarity learning problem and the active similarity learning problem. We refer to our algorithms as OnSim-LinUCB, ActSim-LinUCB,
OnSim-NeuralUCB, ActSim-NeuralUCB, depeding on whether they are based on LinUCB or NeuralUCB and on whether they attempt to solve
the active or online formulation of our problem.

We describe the algorithms by problem.

\subsection{Online similarity learning}
For our problem of online similarity learning we adopt the frameworks of  LinUCB  as described in \cite{linucb} and  NeuralUCB as described in \cite{neuralucb}. We explain this in detail below.

\subsubsection{OnSim-LinUCB}
Our first algorithm performs a straighforward reduction of the online similarity learning problem to that of regular contextual bandits. We do this by assuming a linear structure on the similarity.

Mathematically, if we adopt the formulation we proposed above (\ref{problem-statement:online-similarity-learning}), we choose to model the similarity of two points as  $\phi(\mathbf{x}, \mathbf{y}) = \mathbf{x}^\top \mathbf{A} \mathbf{y}$.
We can see that this is a reasonable thing to do if we consider that
\[\phi(\mathbf{x}, \mathbf{y}) = \mathbf{x}^\top \mathbf{A} \mathbf{y} = \sum_{i =1}^n\sum_{j=1}^n \mathbf{x}_i \mathbf{y}_j \mathbf{A}_{i,j} \]
which we is linear in $\mathbf{A}$ and thus allows us to use the framework of LinUCB for our problem.
Attending to this fact, one can see that Algorithm \ref{algo:onsim-linucb} is almost identical to the first algorithms in \cite{linucb}.
\begin{algorithm}
  \label{algo:onsim-linucb}
  \setstretch{1.2}
    \SetKwInOut{Input}{Input}
    \SetKwInOut{Output}{Output}
    \underline{OnSim-LinUCB} $(a,b)$\;
    \Input{Rounds $T$ and exploration parameter $\alpha$}
    $A \gets I_{n^2}$\;
    $b \gets 0_{n^2}$\;
    \For{$t \in [T]$}{
      $\theta_t \gets A^{-1}b$
      Observe $K$ pairs of vectors $x^k \in R^n$, $y^k \in R^n$\;
      Create $z_{t,k} = (x_1^ky_1^k, x_1^ky_1^k, \dots, x_n^k y_{n-1}^k, x_n^k y_n^k)$\;
      \For{$k \in [K]$}{
        $p_{t,a} \gets \theta_t^\top z_{t,k} + \alpha \sqrt{z_{t,k} A^{-1} x_{t,a}}$
      }
      Choose action $a_t = \text{argmax}_ap_{t,a}$ with ties broken arbitrarily.
      Observe payoff $ r_t \in \{0,1 \}$
      $A \gets A + z_{t,a_t}z_{t,a_t}^\top$\;
      $b \gets z_{t,a_t}r_t$;\
    }
    \caption{OnSim-LinUCB}
\end{algorithm}

\subsubsection{OnSim-NeuralUCB}
Unsurprisingly, the expressive power of the above framework is quite limited.
This limitation comes from the fact that


\section{Experiments}

\section{Conclusion}


\bibliography{refs}
\bibliographystyle{plain}

\end{document}
